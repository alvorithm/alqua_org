\documentclass[spanish,a4paper,12pt,draft]{report}
\usepackage{babel}
\usepackage[latin1]{inputenc}
\usepackage[T1]{fontenc}
\usepackage{eurosym}
\usepackage{amstext}
\usepackage{amscd}
\usepackage{amsmath}
\usepackage{amsfonts}
\usepackage{amssymb}
\usepackage{indentfirst}      % Sangr�a *tambi�n* en el 1er p�rrafo de la secci�n
\usepackage[all]{xy}          % Diagramas conmutativos
\usepackage[mathcal]{eucal}   % Caligr�fica matem�tica
\usepackage{enumerate}



%%%%%%%%%%%%%%%%%%%%%%Teoremas
\newtheorem{teo}{Teorema}[chapter]%%%%%%%%% Teorema
\newtheorem{defi}{Definici�n}[chapter]%%%%%%%% Definicion
\newtheorem{lema}[teo]{Lema}%%%%%%%%%%%%% Lema
\newtheorem{propo}[teo]{Proposici�n}%%%%%%%% Proposicion
\newtheorem{cor}[teo]{Corolario}%%%%%%%%%%%Corolario
\newtheorem{pro1}{}%[chapter]%%%%%%%%%Problema
\newenvironment{pro}{\begin{pro1} \rm} {\end{pro1}}
%\newtheorem{*pro1}[pro1]{* Problema}%%%%%%%%%%Problema complicado
%\newenvironment{*pro}{\begin{*pro1} \sf} {\end{*pro1}}
%%%%%%%%%%%%%%%%%%%%%%%%%%%%%%%


%%%%%%%%%%%%%%%%%%%Comandos
\newcommand{\dem}{\noindent \textbf{Demostraci�n. }\vspace{0.3 cm}}%%Demostracion
\newcommand{\R}{\mathbb{R}}%%%%%%%%%%%%Numeros reales
\newcommand{\F}{\mathbb{F}}%%%%%%%%%%%%Cuerpo
\newcommand{\C}{\mathbb{C}}%%%%%%%%%%%%Numeros complejos
\newcommand{\Q}{\mathbb{Q}}%%%%%%%%%%%%Numeros racionales
\newcommand{\N}{\mathbb{N}}%%%%%%%%%%%%Numeros naturales
\newcommand{\Z}{\mathbb{Z}}%%%%%%%%%%%%Numeros enteros
\newcommand{\Fp}{\mathbb{F}_p}%%%%%%%%%%%%Cuerpo de p elementos
\newcommand{\g}{\mathfrak{g}}%%%%%%%%%%%%Algebra de Lie del grupo G
\newcommand{\V}{\mathcal{V}}%%%%%%%%%%%%Variedad
\newcommand{\W}{\mathcal{W}}%%%%%%%%%%%%Variedad
\newcommand{\h}{\mathcal{H}}%%%%%%%%%%%%Algebra de Lie del grupo H
\newcommand{\uni}{\mathcal{U}}%%%%%%%%%%%%Algebra envolvente
\newcommand{\fin}{ $\Box $ \vspace{0.4 cm}}
\newcommand{\p}{\mathfrak{p}}%%%%%%%% Ideal primo
\newcommand{\Sp}{\mathrm{Sp}}%%%%%%%% grupo simplectico
\newcommand{\m}{\mathfrak{m}}%%%%%%%% Ideal maximal
\newcommand{\limind}{\lim_{\longrightarrow} } 
\newcommand{\gp}{\mathcal{G'}}%%%%%%%%%%%Algebra del grupo G'
\newcommand{\lto}{\longrightarrow}%%%%%%Simplificacion de la flecha larga
\newcommand{\wa}{\omega_2} %%%%%%%%%%%forma simplectica
\newcommand{\Wa}{\Omega_2} %%%%%%%%%% forma simplectica lineal
\newcommand{\lag}{\lambda_g}%%%%%%%%%%%%Traslacion a la izquierda
\newcommand{\rg}{\rho_g}%%%%%%%%%%%%%%%%Traslacion a la derecha
\newcommand{\Gr}{\boldsymbol{G}}%%%%%%%%%%Recubridor universal
\newcommand{\norma}[1]{\: \parallel #1 \!\parallel\! }%%%Norma de un vector
\newcommand{\abs}[1]{\left|\, #1 \right|}  %%%Valor absoluto 
\newcommand{\Pro}{\mathbb{P}}%%%%%%Espacio proyectivo
\newcommand{\Problemas}{\newpage  \begin{center}{\Huge Problemas}\end{center}}
\newcommand{\Ejemplos}{\vspace{0.5 cm} {\bf Ejemplos}}
\newcommand{\D}{\mathcal{D}}%%%%%%%%%%%%Serie derivada
\newcommand{\central}{\mathcal{C}}%%%%%%%%%%%%Serie central
\renewcommand{\to}{\lto}
\newcommand{\escalar}[2]{\left\langle\, #1,#2\, \right\rangle}  %%%Producto escalar 
%%%%%%%%%%%%%%%%%%%%%%%%%%%


%%%%%%%%%%%%%%%%Operadores
\DeclareMathOperator{\End}{End}%%%%%%%%%%Endomorfismo
\DeclareMathOperator{\Ad}{Ad}%%%%%%%%%%Adjunta
\DeclareMathOperator{\grad}{grad}%%%%%%%%%%Graciente
\DeclareMathOperator{\Dif}{Dif}%%%%%%%%%%Diferenciales
\DeclareMathOperator{\sop}{sop}%%%%%%%%%Soporte
\DeclareMathOperator{\distancia}{d}%%%%%%%%Distancia
\DeclareMathOperator{\sen}{sen}%%%%%%%%%%Seno espa�ol
\DeclareMathOperator{\Der}{Der}%%%%%%%%%%Derivaciones
\DeclareMathOperator{\rang}{rang}%%%%%%%%Rango
\DeclareMathOperator{\Hom}{Hom}%%%%%%Homomorfismos
\DeclareMathOperator{\Ann}{Ann}%%%%%%%Anulador
\DeclareMathOperator{\Img}{Im} %%%%Parte imaginaria
\DeclareMathOperator{\rad}{rad}%%%%%%%%Radical
\DeclareMathOperator{\Ker}{Ker}%%%%%%%Nucleo
\DeclareMathOperator{\Id}{Id}%%%%%%% Identidad
\DeclareMathOperator{\GL}{GL}%%%%%%%%%Grupo lineal
\DeclareMathOperator{\Apli}{Apli}%%%%%%Aplicaciones
\DeclareMathOperator{\Bil}{Bil}%%%%%Bilineales
\DeclareMathOperator{\Spec}{Spec}%%%%Espectro
\DeclareMathOperator{\Aut}{Aut}%%%%Automorfismo
\DeclareMathOperator{\Ob}{Ob}  %%% Objetos de una categor�a
\DeclareMathOperator{\Traza}{Traza}  %%% Traza de un endomorfismo
\DeclareMathOperator{\ad}{ad}  %%% Traza de un endomorfismo
%%%%%%%%%%%%%%%%%%%%%%%%%%%%%%%


\begin{document}
\chapter{Mec�nica lagrangiana}

\section{Introducci�n}

Dada una variedad diferenciable $\V$, consideramos su fibrado tangente $T(\V)$.  Tambi�n lo podemos llamar {\sf espacio de fases de velocidades}. \index{espacio de fases!de velocidades} Tomamos una funci�n diferenciable $L \in C^\infty (T(\V))$, definida sobre el fibrado tangente.  Dicha funci�n recibe el nombre de {\sf lagrangiano}, \index{lagrangiano} por analog�as de la f�sica cl�sica.

Fijados estos elementos podemos construir una funci�n del fibrado tangente en el fibrado cotangente de la variedad.  Es lo que llamaremos {\sf transformaci�n de Legrendre} \index{transformaci�n!de Legendre} asociada al lagrangiano $L$.

En muchos casos la transformaci�n de Legendre establece un difeomorfismo (local), lo que permite introducir una estructura simpl�ctica en el fibrado tangente.

El estudio de las ecuaciones del movimiento, entendiendolas como campos de vectores en el fibrado tangente, es lo que denominamos\index{mec�nica!lagrangiana}
 {\sf mec�nica lagrangiana}. 
Bajo ciertas hip�tesis se puede probar que este enfoque y el enfoque hamiltoniano son equivalentes.

\section{Derivada en las fibras}

Sea $L : T(\V) \rightarrow \R$ una funci�n diferenciable y $x \in \V$ un punto arbitrario.  Los vectores tangentes en ese punto los denotamos $v_x$.  Sea  $L_x$  la funci�n $L$ restringida a $T_x(\V)$.  Esto es
$$
\begin{array}{cccl}
L_x: & T_x(\V) & \rightarrow & \R \\
      &  v_x& \rightarrow & L_x(v_x) = L(x,v_x)
\end{array}
$$

La funci�n $L_x$ est� definida en un espacio vectorial y dado un punto $v_x$ del espacio vectorial  tiene sentido calcular su diferencial $D_{v_x}L_x$. Esta aplicaci�n tiene como dominio el mismo espacio y como imagen $\R$.  De este modo podemos considerarla como un elemento del espacio dual, $T^*_x(\V)$.

\begin{defi}

Llamamos {\sf transformaci�n de Legendre} \index{transformaci�n!de Legendre} (asociada al lagrangiano $L$) a la aplicaci�n

$$
\begin{array}{cccl}
\mathbf{L} : & T(\V) &\rightarrow&  T^*(\V) \\
            &   v_x  & \rightarrow & D_{v_x}L_x
\end{array}
$$

\end{defi}

Viendo la aplicaci�n como una derivada tenemos
$$
\mathbf{L}(v_x)(v'_x) = {\left ( \frac{d}{ds}\right )}_{s=0}L(v_x+s v'_x)
$$

Dicha funci�n transforma vectores tangentes en un punto, en vectores cotangentes en ese mismo punto.  Tenemos el diagrama conmutativo
$$
\xymatrix{
T(\V)\ar[rd] \ar[rr]^{\mathbf{L}} &  & T^(\V)\ar[ld] \\
    & \V & }
$$



\section{Lagrangianos regulares}

\begin{defi}

Un lagrangiano es {\sf regular}  \index{lagrangiano!regular} si $\mathbf{L}$  difeomorfismo local.
Un lagrangiano es {\sf hiperregular} \index{lagrangiano!hiperregular} si $\mathbf{L}$ es un difeomorfismo.

\end{defi}

Todo lagrangiano hiperregular es tambi�n regular.  Supondremos a partir de ahora que todo lagrangiano es regular.

\begin{defi}

La $1$-forma $\theta_L= \mathbf{L}^*(\theta)$ se llama {\sf $1$-forma de Lagrange}. \index{$1$-forma de Lagrange}
La $2$ forma $\omega_L= \mathbf{L}^*(\wa)$ se llama {\sf $2$-forma de Lagrange}.  \index{$2$-forma de Lagrange}

\end{defi}

Como la diferencial exterior conmuta con la imagen inversa tenemos que $\omega_L= d \theta_L  $.  En particular,  la $2$-forma de Lagrange es cerrada.  

Esto es v�lido para todos los lagrangianos.  Pero en el caso regular tenemos

\begin{propo}

Si el lagrangiano es regular, la forma de Lagrange dota al fibrado tangente de una estructura de variedad simpl�ctica.

\end{propo}

\dem

Ya vimos que la forma era cerrada.  Si el lagrangiano es regular, la aplicaci�n tangente en todo punto es un isomorfismo y por lo tanto, punto a punto la forma de Lagrange es no degenerada, debido  a que es la imagen por un isomorfismo de una forma no degenerada. \fin

Si la $2$-forma de Lagrange es simpl�ctica, en cada punto la transformaci�n de Lagrange tiene que tener como diferencial un isomorfismo.  De este modo, la transformaci�n de Lagrange ser� un difeormorsimo local en todo punto.

Esto nos conduce a una definici�n m�s cl�sica de la regularidad de un lagrangiano.

\begin{defi}

Un lagrangiano es {\sf regular} \index{lagrangiano!regular} si la $2$-forma de Lagrange asociada es una forma simpl�ctica.

\end{defi}

\section{Ecuaciones de Euler-Lagrange}

\begin{defi}

Llamamos {\sf acci�n} \index{acci�n} de un lagrangiano a la funci�n
$$
\begin{array}{cccc}
A : & T(\V) &  \rightarrow & \R  \\
    &  v_x & \rightarrow &\mathbf{L}(v_x)(v_x)
\end{array}
$$
  La {\sf energ�a} \index{energ�a} es la funci�n $E= A-L$.

\end{defi}

Como estamos suponiendo que el lagrangiano es regular, la funci�n energia tiene asociado un campo hamiltoniano, $X_E$, definido en el fibrado tangente.  Llamaremos a ese campo, {\sf campo lagrangiano}.
\index{campo!lagrangiano}



\end{document}