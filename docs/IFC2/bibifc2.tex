\documentclass[spanish,a4paper,12pt]{book}
\usepackage{babel}
\usepackage[latin1]{inputenc}
\usepackage[T1]{fontenc}
\usepackage{eurosym}
\usepackage[squaren]{SIunits}
\usepackage{longtable}
\usepackage{amsmath}
\usepackage{amssymb}
\usepackage{subfigure}
\usepackage{float}
\usepackage{verbatim}
\newcommand{\noun}[1]{\textsc{#1}}
%Because html converters don't know tabularnewline
\providecommand{\tabularnewline}{\\}
\newcommand{\lyxline}[1]{
  {#1 \vspace{1ex} \hrule width \columnwidth \vspace{1ex}}
}
\makeatletter
\providecommand{\LyX}{L\kern-.1667em\lower.25em\hbox{Y}\kern-.125emX\@}

%% Special footnote code from the package 'stblftnt.sty'
%% Author: Robin Fairbairns -- Last revised Dec 13 1996
\let\SF@@footnote\footnote
\def\footnote{\ifx\protect\@typeset@protect
    \expandafter\SF@@footnote
  \else
    \expandafter\SF@gobble@opt
  \fi
}
\expandafter\def\csname SF@gobble@opt \endcsname{\@ifnextchar[%]
  \SF@gobble@twobracket
  \@gobble
}
\edef\SF@gobble@opt{\noexpand\protect
  \expandafter\noexpand\csname SF@gobble@opt \endcsname}
\def\SF@gobble@twobracket[#1]#2{}

\floatstyle{ruled}
\newfloat{algorithm}{tbp}{loa}
\floatname{algorithm}{Algoritmo}

%%%%%%%%%%%%%%%%%%%%%%%%%%%%%% Textclass specific LaTeX commands.
 \newenvironment{lyxcode}
   {\begin{list}{}{
     \setlength{\rightmargin}{\leftmargin}
     \setlength{\listparindent}{0pt}% needed for AMS classes
     \raggedright
     \setlength{\itemsep}{0pt}
     \setlength{\parsep}{0pt}
     \normalfont\ttfamily}%
    \item[]}
   {\end{list}}


\addto\extrasspanish{\bbl@deactivate{~}}
\makeatother

\usepackage[dvips,dvipdfm]{graphicx}
\usepackage{multicol}


\input{IFC2-macros.sty}
\begin{document}

\chapter{Comentario a la bibliograf�a.}

El \cite{Sanchez} es un libro de un nivel similar o, ligeramente
superior al del curso. Trata muchos temas previos a este curso y algunos
posteriores. El \cite{Gasiorowicz} es el libro que m�s se aproxima
a la parte de f�sica at�mica de este curso. El \cite{Alonso} es �til
para la segunda parte del curso, dedicada a las estad�sticas cu�nticas.
El \cite{Cohen} es un libro de mec�nica cu�ntica bastante amplio
y completo que trata en ap�ndices numerosos problemas de f�sica cu�ntica.
Tiene un nivel superior al de este curso. Cabe destacar las poderosas
analog�as que los autores son capaces de establecer con la �ptica.
Por �ltimo, el \cite{Ballentine} es otro libro excelente que conjuga
de forma bastante acertada formalismo e interpretaci�n de la mec�nica
cu�ntica. Nivel superior al del curso, pero completamente recomendable
para un estudio serio de la materia.

\begin{thebibliography}{Gasiorowicz}
\bibitem[Sanchez]{Sanchez}\noun{S�nchez} \noun{del} \noun{R�o}, C. (coord.): \emph{F�sica
Cu�ntica.} Ed. Pir�mide, 1997. \\

\bibitem[Gasiorowicz]{Gasiorowicz}\noun{Gasiorowicz}, S.: \emph{Quantum Physics.} Ed. John Wiley \&
Sons, 1996. \\

\bibitem[Alonso]{Alonso}\noun{Alonso}, M. y \noun{Finn}, E.: \emph{F�sica Cu�ntica (vol
III)}. Ed. Fondo educativo interamericano, 1971. \\

\bibitem[Cohen]{Cohen}\noun{Cohen}--\noun{Tannoudji}, C., \noun{Diu}, B. y \noun{Lalo�},
F.: \emph{M�canique quantique.} Vols. I y II. Hermann, Paris. \\

\bibitem[Ballentine]{Ballentine}\noun{Ballentine, L.} \emph{Quantum Mecanics, a Modern Development.}
Ed. World Scientific, 2000.\\

\bibitem{key-2}\noun{Woodgate, G. K.}: \emph{Elementary atomic structure}, 2nd
ed. Ed. Oxford Science Publications, 1998.\\
\end{thebibliography}

\end{document}

